\documentclass[
    % -- opções da classe memoir --
    12pt,                % tamanho da fonte
    openright,            % capítulos começam em pág ímpar (insere página vazia caso preciso)
    twoside,            % para impressão em recto e verso. Oposto a oneside
    a4paper,            % tamanho do papel. 
    % -- opções da classe abntex2 --
    chapter=TITLE,        % títulos de capítulos convertidos em letras maiúsculas
    %section=TITLE,        % títulos de seções convertidos em letras maiúsculas
    %subsection=TITLE,    % títulos de subseções convertidos em letras maiúsculas
    %subsubsection=TITLE,% títulos de subsubseções convertidos em letras maiúsculas
    % -- opções do pacote babel --
    english,            % idioma adicional para hifenização
    french,                % idioma adicional para hifenização
    spanish,            % idioma adicional para hifenização
    brazil                % o último idioma é o principal do documento
    ]{abntex2}

% Tudo é UTF-8.
\usepackage[utf8]{inputenc}


% ---
% Informações de dados para CAPA e FOLHA DE ROSTO
% ---
\titulo{Solucionando Restrições Lógicas em GPUs}
\autor{Flávio Lisbôa da C. Costa}
\local{Universidade Veiga de Almeida\\Rio de Janeiro, Brasil}
\data{2016}
\orientador{(Quem será o meu orientador?)}
\coorientador{(Sem co-orientador...)}
\instituicao{
  Universidade Veiga de Almeida -- UVA
  \par
  Bacharelado em Ciência da Computação
}
\tipotrabalho{Monografia}
\preambulo{
Monografia apresentada à banca examinadora do curso de Ciência da Computação da Universidade Veiga de Almeida, como requisito parcial para obtenção do título de Bacharel em Ciência da Computação.
}
% ---


% ---
% Pacotes e Estilos
% ---
\usepackage{mono}
% ---


% ---
% informações do PDF
% ---
\makeatletter
\hypersetup{
        %pagebackref=true,
        pdftitle={\@title}, 
        pdfauthor={\@author},
        pdfsubject={\imprimirpreambulo},
        pdfcreator={LaTeX with abnTeX2},
        pdfkeywords={Interpretação Abstrata}{Domínio dos octógonos}{GPGPU}{Computação Paralela}, 
        colorlinks=false,               % false: boxed links; true: colored links
        linkcolor=blue,              % color of internal links
        citecolor=blue,                % color of links to bibliography
        filecolor=magenta,              % color of file links
        urlcolor=blue,
        bookmarksdepth=4
}
\makeatother
% ---


\makeindex
\makeglossaries

\loadglsentries{./tex/preambulo-glossario.tex}

% ----
% Início do documento
% ----
\begin{document}


% Seleciona o idioma do documento (conforme pacotes do babel)
%\selectlanguage{english}
\selectlanguage{brazil}


% ----------------------------------------------------------
% ELEMENTOS PRÉ-TEXTUAIS
% ----------------------------------------------------------
%\pretextual
\include{./tex/pre-01.00-capa}
% ---
% Folha de rosto
% (o * indica que haverá a ficha bibliográfica)
% ---
\imprimirfolhaderosto*
% ---

% ---
% Anverso da folha de rosto:
% ---

{
\ABNTEXchapterfont

\vspace*{\fill}
\input{./tex/pre-02.01-ficha-bibliografica}

\vspace*{\fill}
}

%\include{./tex/pre-03.00-ficha-bibliografica}
%\include{./tex/pre-04.00-errata.tex}
\include{./tex/pre-05.00-folha-aprovacao}

% ---
% Dedicatória
% ---
\begin{dedicatoria}
    \vspace*{\fill}
    \hspace{.45\textwidth}
    \begin{minipage}{.5\textwidth}
        \centering
        \noindent
        \textit{ Este trabalho é dedicado às crianças adultas que, quando pequenas, sonharam em se tornar cientistas.}
    \end{minipage}%
\end{dedicatoria}
% ---


% ---
% Agradecimentos
% ---
\begin{agradecimentos}

% TODO Agradecimentos
Agradeço à minha mãe.

\end{agradecimentos}
% ---

\include{./tex/pre-08.00-epigrafe}

% ---
% RESUMOS
% ---

% resumo em português
\setlength{\absparsep}{18pt} % ajusta o espaçamento dos parágrafos do resumo
\begin{resumo}
 Este é o resumo.

 \textbf{Palavras-chave}: latex. abntex. editoração de texto.
\end{resumo}

% resumo em inglês
\begin{resumo}[Abstract]
 \begin{otherlanguage*}{english}
   This is the english abstract.

   \vspace{\onelineskip}
 
   \noindent 
   \textbf{Keywords}: latex. abntex. text editoration.
 \end{otherlanguage*}
\end{resumo}


\include{./tex/pre-10.00-listas}
\include{./tex/pre-11.00-sumario}


% ----------------------------------------------------------
% ELEMENTOS TEXTUAIS
% ----------------------------------------------------------
\textual


\chapter[Introdução]{Introdução}

    Citação normal: \cite{chawdhary_simple_2014}. Citação online: \citeonline{chawdhary_simple_2014}.
\chapter{Interpretação Abstrata}
    \section{Conceitos Básicos}
    \section{Domínios Abstratos}
        \subsection{Domínios não-relacionais}
        \subsection{Domínios Relacionais}
    \section{Domínio dos Octógonos}
        \subsection{Vantagens e Diferenciais}
        \subsection{Matrizes de Diferenças}
        \subsection{Operadores}
\chapter{Computação Paralela}
    \section{OpenCL}
        \subsection{Arquitetura Geral}
        \subsection{Workgroups, Work-items e Memória}
        \subsection{Exemplos}
        

\chapter{Metodologia}

\section{Vestibulum ante ipsum primis in faucibus orci luctus et ultrices
posuere cubilia Curae}

\lipsum[21-22]

\section{Pellentesque sit amet pede ac sem eleifend consectetuer}

\lipsum[24]

\include{./tex/textual-04.00-desenvolvimento}
\include{./tex/textual-05.00-resultados}

\phantompart
\chapter{Conclusão}

\lipsum[31-33]



% ----------------------------------------------------------
% ELEMENTOS PÓS-TEXTUAIS
% ----------------------------------------------------------
\postextual

% ----------------------------------------------------------
% Referências bibliográficas
% ----------------------------------------------------------
%\addbibresource{mono.bib}
\bibliography{mono}



\include{./tex/pos-02.00-glossario}

% ----------------------------------------------------------
% Apêndices
% ----------------------------------------------------------

% ---
% Inicia os apêndices
% ---
\begin{apendicesenv}


\end{apendicesenv}
% ---



% ----------------------------------------------------------
% Anexos
% ----------------------------------------------------------

% ---
% Inicia os anexos
% ---
\begin{anexosenv}

% Imprime uma página indicando o início dos anexos
%\partanexos

% ---
\chapter{Anexo qualquer.}
% ---
\lipsum[30]

% ---
\chapter{Cras non urna sed feugiat cum sociis natoque penatibus et magnis dis
parturient montes nascetur ridiculus mus}
% ---

\lipsum[31]

% ---
\chapter{Fusce facilisis lacinia dui}
% ---

\lipsum[32]

\end{anexosenv}



%---------------------------------------------------------------------
% INDICE REMISSIVO
%---------------------------------------------------------------------
\phantompart
\printindex
%---------------------------------------------------------------------

\listoftodos

\end{document}
% ----
