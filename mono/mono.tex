\documentclass[
    % -- opções da classe memoir --
    12pt,                % tamanho da fonte
    openright,            % capítulos começam em pág ímpar (insere página vazia caso preciso)
    twoside,            % para impressão em recto e verso. Oposto a oneside
    a4paper,            % tamanho do papel. 
    % -- opções da classe abntex2 --
    chapter=TITLE,        % títulos de capítulos convertidos em letras maiúsculas
    %section=TITLE,        % títulos de seções convertidos em letras maiúsculas
    %subsection=TITLE,    % títulos de subseções convertidos em letras maiúsculas
    %subsubsection=TITLE,% títulos de subsubseções convertidos em letras maiúsculas
    % -- opções do pacote babel --
    english,            % idioma adicional para hifenização
    french,                % idioma adicional para hifenização
    spanish,            % idioma adicional para hifenização
    brazil                % o último idioma é o principal do documento
    ]{abntex2}

% Tudo é UTF-8.
\usepackage[utf8]{inputenc}


% ---
% Informações de dados para CAPA e FOLHA DE ROSTO
% ---
\titulo{Solucionando Restrições Lógicas em GPUs}
\autor{Flávio Lisbôa da C. Costa}
\local{Universidade Veiga de Almeida\\Rio de Janeiro, Brasil}
\data{2016}
\orientador{(Quem será o meu orientador?)}
\coorientador{(Sem co-orientador...)}
\instituicao{
  Universidade Veiga de Almeida -- UVA
  \par
  Bacharelado em Ciência da Computação
}
\tipotrabalho{Monografia}
\preambulo{
Monografia apresentada à banca examinadora do curso de Ciência da Computação da Universidade Veiga de Almeida, como requisito parcial para obtenção do título de Bacharel em Ciência da Computação.
}
% ---


% ---
% Pacotes e Estilos
% ---
\usepackage{mono}
% ---


% ---
% informações do PDF
% ---
\makeatletter
\hypersetup{
        %pagebackref=true,
        pdftitle={\@title}, 
        pdfauthor={\@author},
        pdfsubject={\imprimirpreambulo},
        pdfcreator={LaTeX with abnTeX2},
        pdfkeywords={Interpretação Abstrata}{Domínio dos octógonos}{GPGPU}{Computação Paralela}, 
        colorlinks=false,               % false: boxed links; true: colored links
        linkcolor=blue,              % color of internal links
        citecolor=blue,                % color of links to bibliography
        filecolor=magenta,              % color of file links
        urlcolor=blue,
        bookmarksdepth=4
}
\makeatother
% ---


\makeindex
\makeglossaries

%\loadglsentries{./tex/preambulo-glossario.tex}

% ---
% S Í M B O L O S
% ---

\novosimbolo
    {mypi} %label
    {\ensuremath{\pi}} %symbol
    {Pi} %name
    {Razão entre a circunferência de um círculo e o seu diâmetro.} %brief


% ---
% A C R Ô N I M O S
% ---

\novoacronimo
    {IA} %label
    {IA} %symbol
    {{Interpretação Abstrata}} %name
    {
        A Interpretação Abstrata, geralmente abreviada como IA (ou AI na 
        literatura internacional), é uma teoria que formaliza aproximações 
        plausíveis da semântica de programas e sistemas computacionais. Mais 
        especificamente, trata do problema de obter informações sobre a 
        semântica de programas através de execuções parciais.
    }

\novoacronimo
    {TI}
    {TI}
    {{Tecnologia da Informação}}
    {
        Em uma forma mais ampla, pode-se definir a Tecnologia da 
        Informação 
        como o conjunto de técnicas, atividades e soluções voltadas à 
        gerência, 
        geração, uso, acesso, segurança e transmissão de dados e 
        informação por 
        meios computacionais.
    }

\novoacronimo
    {STR}
    {STR}
    {{Sistema de Tempo Real}}
    {
        Sistemas nos quais há exigências temporais acerca da realização de suas 
        tarefas. Não se trata de executar as tarefas no menor tempo possível, 
        mas sim de executá-las \emph{dentro do tempo disponível}.
    }

% ---
% D E F I N I Ç Õ E S
% ---

\novadefinicaopl
    {sistema_critico}
    {sistema crítico}
    {sistemas críticos}
    {
        Vide seção \emph{\ref{introducao_sobre_criticidade}} (página 
        \pageref{introducao_sobre_criticidade}).
    }

\novadefinicao
    {dependabilidade}
    {dependabilidade}
    {
        Termo que representa uma abordagem mais abrangente, em relação à 
        caracterização da criticidade de sistemas. Explicando de forma sucinta, 
        entende-se que a \emph{dependabilidade} de um sistema não 
        seja definida apenas pelos seus aspectos técnicos, assim como 
        ocorre com termos como \emph{confiabilidade}, muito usados 
        hoje. \citeauthor{rushby_critical_1994} cita o trabalho de 
        \citeonline{laprie_dependable_1985} para definir o conceito 
        de \emph{dependabilidade}:
        \blockquote{\textit{
            O termo \emph{"dependabilidade"} é usado para escapar de 
            significados técnicos especializados que hoje estão 
            associados à termos como \emph{"confiabilidade"}. Também almeja-se 
            ter um termo para uma abordagem que, no geral, possa adotar, 
            incluir e unificar muitos problemas e técnicas que têm geralmente 
            sido consideradas isoladamente -- como por exemplo, tolerância a 
            falhas, confiabilidade, corretude, proteção, capacidade de 
            sobrevivência e segurança.} \cite{rushby_critical_1994}
        }
    }

\novadefinicao
    {protecao}
    {proteção contra falhas}
    {
        O termo original em inglês é \emph{safety}, que significa 
        \emph{segurança}, se traduzido literalmente para o português. No 
        entanto, há uma conotação forte do termo \emph{segurança} para questões 
        que envolvem a \emph{segurança da informação}. Na literatura estudada, 
        \emph{safety} possui um sentido diferente, mais voltado à 
        \emph{segurança contra falhas}; sobre garantias de integridade do 
        sistema, de pessoas, do meio ambiente, de propriedades e de ativos após 
        a ocorrência de uma falha.
        
        Infelizmente, não foi encontrado nenhum termo na literatura brasileira 
        com este significado específico. O uso desta expressão fica, portanto, 
        exclusiva à esta monografia, apenas para explicitar a diferença em 
        relação às terminologias que envolvem segurança da informação.
    }


% ----
% Início do documento
% ----
\begin{document}


% Seleciona o idioma do documento (conforme pacotes do babel)
%\selectlanguage{english}
\selectlanguage{brazil}


% ----------------------------------------------------------
% ELEMENTOS PRÉ-TEXTUAIS
% ----------------------------------------------------------
%\pretextual
% ---
% Capa
% ---
\imprimircapa
% ---

% ---
% Folha de rosto
% (o * indica que haverá a ficha bibliográfica)
% ---
\imprimirfolhaderosto*
% ---

% ---
% Anverso da folha de rosto:
% ---

{
\ABNTEXchapterfont

\vspace*{\fill}

% ---
% Inserir a ficha bibliografica
% ---

% Isto é um exemplo de Ficha Catalográfica, ou ``Dados internacionais de
% catalogação-na-publicação''. Você pode utilizar este modelo como referência. 
% Porém, provavelmente a biblioteca da sua universidade lhe fornecerá um PDF
% com a ficha catalográfica definitiva após a defesa do trabalho. Quando estiver
% com o documento, salve-o como PDF no diretório do seu projeto e substitua todo
% o conteúdo de implementação deste arquivo pelo comando abaixo:
%
% \begin{fichacatalografica}
%     \includepdf{fig_ficha_catalografica.pdf}
% \end{fichacatalografica}

% Opcionalmente...
%

\begin{fichacatalografica}
	\sffamily
	\vspace*{\fill}					% Posição vertical
	\begin{center}					% Minipage Centralizado
	\fbox{\begin{minipage}[c][8cm]{13.5cm}		% Largura
	\small
	\imprimirautor
	%Sobrenome, Nome do autor
	
	\hspace{0.5cm} \imprimirtitulo  / \imprimirautor. --
	\imprimirlocal, \imprimirdata-
	
	\hspace{0.5cm} \pageref{LastPage} p. : il. (algumas color.) ; 30 cm.\\
	
	\hspace{0.5cm} \imprimirorientadorRotulo~\imprimirorientador\\
	
	\hspace{0.5cm}
	\parbox[t]{\textwidth}{\imprimirtipotrabalho~--~\imprimirinstituicao,
	\imprimirdata.}\\
	
	\hspace{0.5cm}
		1. Palavra-chave1.
		2. Palavra-chave2.
		2. Palavra-chave3.
		I. Orientador.
		II. Universidade xxx.
		III. Faculdade de xxx.
		IV. Título 			
	\end{minipage}}
	\end{center}
\end{fichacatalografica}
% ---


\vspace*{\fill}
}

%\include{./tex/pre-03.00-ficha-bibliografica}
%
% ---
% Inserir errata
% ---
\begin{errata}
Elemento opcional da \citeonline[4.2.1.2]{NBR14724:2011}. Exemplo:

\vspace{\onelineskip}

FERRIGNO, C. R. A. \textbf{Tratamento de neoplasias ósseas apendiculares com
reimplantação de enxerto ósseo autólogo autoclavado associado ao plasma
rico em plaquetas}: estudo crítico na cirurgia de preservação de membro em
cães. 2011. 128 f. Tese (Livre-Docência) - Faculdade de Medicina Veterinária e
Zootecnia, Universidade de São Paulo, São Paulo, 2011.

\begin{table}[htb]
\center
\footnotesize
\begin{tabular}{|p{1.4cm}|p{1cm}|p{3cm}|p{3cm}|}
  \hline
   \textbf{Folha} & \textbf{Linha}  & \textbf{Onde se lê}  & \textbf{Leia-se}  \\
    \hline
    1 & 10 & auto-conclavo & autoconclavo\\
   \hline
\end{tabular}
\end{table}

\end{errata}
% ---



% ---
% Inserir folha de aprovação
% ---

% Isto é um exemplo de Folha de aprovação, elemento obrigatório da NBR
% 14724/2011 (seção 4.2.1.3). Você pode utilizar este modelo até a aprovação
% do trabalho. Após isso, substitua todo o conteúdo deste arquivo por uma
% imagem da página assinada pela banca com o comando abaixo:
%
% \includepdf{folhadeaprovacao_final.pdf}
%
\begin{folhadeaprovacao}

  \begin{center}
    {\ABNTEXchapterfont\large\imprimirautor}

    \vspace*{\fill}\vspace*{\fill}
    \begin{center}
      \ABNTEXchapterfont\bfseries\Large\imprimirtitulo
    \end{center}
    \vspace*{\fill}
    
    \hspace{.45\textwidth}
    \begin{minipage}{.5\textwidth}
        \imprimirpreambulo
    \end{minipage}%
    \vspace*{\fill}
   \end{center}
        
   Trabalho aprovado. \imprimirlocal, 24 de novembro de 2012:

   \assinatura{\textbf{\imprimirorientador} \\ Orientador} 
   \assinatura{\textbf{Professor} \\ Convidado 1}
   \assinatura{\textbf{Professor} \\ Convidado 2}
   %\assinatura{\textbf{Professor} \\ Convidado 3}
   %\assinatura{\textbf{Professor} \\ Convidado 4}
      
   \begin{center}
    \vspace*{0.5cm}
    {\large\imprimirlocal}
    \par
    {\large\imprimirdata}
    \vspace*{1cm}
  \end{center}
  
\end{folhadeaprovacao}
% ---


% ---
% Dedicatória
% ---
\begin{dedicatoria}
   \vspace*{\fill}
   \centering
   \noindent
   \textit{ Este trabalho é dedicado às crianças adultas que,\\
   quando pequenas, sonharam em se tornar cientistas.} \vspace*{\fill}
\end{dedicatoria}
% ---


% ---
% Agradecimentos
% ---
\begin{agradecimentos}

% TODO Agradecimentos
Agradeço à minha mãe.

\end{agradecimentos}
% ---


% ---
% Epígrafe
% ---
\begin{epigrafe}
    \vspace*{\fill}
	\begin{flushright}
		\textit{``Não vos amoldeis às estruturas deste mundo, \\
		mas transformai-vos pela renovação da mente, \\
		a fim de distinguir qual é a vontade de Deus: \\
		o que é bom, o que Lhe é agradável, o que é perfeito.\\
		(Bíblia Sagrada, Romanos 12, 2)}
	\end{flushright}
\end{epigrafe}
% ---


% ---
% RESUMOS
% ---

% resumo em português
\setlength{\absparsep}{18pt} % ajusta o espaçamento dos parágrafos do resumo
\begin{resumo}

O campo de Interpretação Abstrata (IA) avançou muito desde a sua concepção nos 
anos 70. A IA é usada para aproximar o comportamento e modelar sistemas formais 
e computacionais. Ferramentas baseadas em conceitos de IA são usadas 
constantemente em \emph{frameworks} de análise estática de programas, para 
realizar tarefas diversas como prever defeitos, provar a corretura ou ausência 
de bugs, encontrar falhas de segurança e inferir tipos de dados em ambientes de 
linguagens de programação. Um conceito crucial na área de IA é o do domínio 
abstrato. Muitos domínios numéricos foram criados e usados com sucesso para 
resolver problemas reais, como os domínios dos intervalos, das congruências e 
dos polígonos.

Um deles, no entanto, tem recebido destaque recentemente: o domínio dos 
octógonos. Criado em 2006 por Antoine Miné, é um domínio abstrato fracamente 
relacional, que merece destaque pela sua baixa complexidade computacional e 
pela razoável adequação do algoritmo ao paralelismo. Analisadores estáticos já 
foram criados com base neste domínio abstrato, e usados com sucesso, mas poucos 
deles até hoje exploraram a possibilidade de implementar os operadores 
octogonais de forma paralela em GPU. Considerando o custo cada vez mais baixo 
de placas de vídeo e de plataformas de GPGPU (General Processing GPU), torna-se 
relevante a exploração de possíveis implementações e adaptações dos algoritmos 
dos domínios dos octógonos para esta nova realidade.

O objetivo deste trabalho é implementar os algoritmos e operadores do domínio 
dos octógonos em GPU, utilizando o OpenCL. Comparações serão realizadas com 
outras duas implementações baseadas em CPU e SSE, a fim de concluir a adequação 
dos algoritmos e decidir em quais tipos de ocasião o uso de computação paralela 
em GPU se torna mais apropriado.

\vspace{\onelineskip}

% TODO: Criar um comando \keywords para as palavras-chave.
\textbf{Palavras-chave}: Interpretação Abstrata. Domínio dos octógonos. GPGPU. Computação Paralela.
\end{resumo}

% resumo em inglês
\begin{resumo}[Abstract]
 \begin{otherlanguage*}{english}
   This is the english abstract.

   \vspace{\onelineskip}
 
   \noindent 
   \textbf{Keywords}: latex. abntex. text editoration.
 \end{otherlanguage*}
\end{resumo}




% ---
% inserir lista de ilustrações
% ---
\pdfbookmark[0]{\listfigurename}{lof}
\listoffigures*
\cleardoublepage
% ---

% ---
% inserir lista de tabelas
% ---
\pdfbookmark[0]{\listtablename}{lot}
\listoftables*
\cleardoublepage
% ---


% ---
% inserir lista de abreviaturas e siglas
% ---
\begin{siglas}
  \item[ABNT] Associação Brasileira de Normas Técnicas
  \item[abnTeX] ABsurdas Normas para TeX
\end{siglas}
% ---


% ---
% inserir lista de símbolos
% ---
\begin{simbolos}
  \item[$ \Gamma $] Letra grega Gama
  \item[$ \Lambda $] Lambda
  \item[$ \zeta $] Letra grega minúscula zeta
  \item[$ \in $] Pertence
\end{simbolos}
% ---

% ---
% inserir o sumario
% ---
\pdfbookmark[0]{\contentsname}{toc}
\tableofcontents*
\cleardoublepage
% ---




% ----------------------------------------------------------
% ELEMENTOS TEXTUAIS
% ----------------------------------------------------------
\textual


\chapter[Introdução]{Introdução}

    Citação normal: \cite{chawdhary_simple_2014}. Citação online: \citeonline{chawdhary_simple_2014}.
\chapter{Interpretação Abstrata} \label{fundamentacao_ia}
    \section{Conceitos Básicos}
    \section{Domínios Abstratos}
        \subsection{Domínios não-relacionais}
        \subsection{Domínios Relacionais}
    \section{Domínio dos Octógonos}
        \subsection{Vantagens e Diferenciais}
        \subsection{Matrizes de Diferenças}
        \subsection{Operadores}


\chapter{Computação Paralela} \label{fundamentacao_cl}
    \section{Conceitos gerais}
    \section{OpenCL}
        \subsection{Arquitetura Geral}
        \subsection{Workgroups, Work-items e Memória}
        \subsection{Exemplos}
        

\chapter{Metodologia} \label{metodologia}


\chapter{Desenvolvimento}


\chapter{Resultados} \label{resultados}


\phantompart
\chapter{Conclusão}



% ----------------------------------------------------------
% ELEMENTOS PÓS-TEXTUAIS
% ----------------------------------------------------------
\postextual

% ----------------------------------------------------------
% Referências bibliográficas
% ----------------------------------------------------------
%\bibliography{../mono}


% ----------------------------------------------------------
% Glossário
% ----------------------------------------------------------
%
% Consulte o manual da classe abntex2 para orientações sobre o glossário.
%
%\glossary



% ----------------------------------------------------------
% Apêndices
% ----------------------------------------------------------

% ---
% Inicia os apêndices
% ---
\begin{apendicesenv}

% Imprime uma página indicando o início dos apêndices
%\partapendices

% ----------------------------------------------------------
\chapter{Apêndice qualquer}
% ----------------------------------------------------------

\lipsum[50]

% ----------------------------------------------------------
\chapter{Nullam elementum urna vel imperdiet sodales elit ipsum pharetra ligula
ac pretium ante justo a nulla curabitur tristique arcu eu metus}
% ----------------------------------------------------------
\lipsum[55-57]

\end{apendicesenv}
% ---



% ----------------------------------------------------------
% Anexos
% ----------------------------------------------------------

% ---
% Inicia os anexos
% ---
\begin{anexosenv}


\end{anexosenv}



%---------------------------------------------------------------------
% INDICE REMISSIVO
%---------------------------------------------------------------------
\phantompart
\printindex
%---------------------------------------------------------------------



\end{document}
% ----
