
% ---
% S Í M B O L O S
% ---

\novosimbolo
    {rel} %label
    {\ensuremath{\bowtie}} %symbol
    {rel} %name
    {
        Um comparador ou operador de relação entre dois elementos quaisquer de 
        um conjunto.
    } %brief


% ---
% A C R Ô N I M O S
% ---

\novoacronimo
    {CPU}
    {CPU}
    {{Central Processing Unit}}
    {
        Unidade Central de Processamento. Em um computador, é responsável pelo 
        processamento principal. Programas são inicialmente executados na CPU, 
        que pode ativar o funcionamento ou delegar tarefas para outras unidades 
        de trabalho na arquitetura computacional. Difere de outras unidades com 
        funcionalidades específicas, como a \gls{GPU}.
    }

\novoacronimo
    {GPU}
    {GPU}
    {{Graphics Processing Unit}}
    {
        Unidade de Processamento Gráfico. Em um computador, é responsável pela 
        renderização de gráficos em tempo real para um dispositivo de saída, 
        como monitores.
    }
    
    
\novoacronimo
    {TDD}
    {TDD}
    {{Test-Driven Development}}
    {
        Desenvolvimento orientado a testes.
        
        \TODO{Melhor descrição para TDD.}
    }
    
\novoacronimo
    {BDD}
    {BDD}
    {{Behaviour-Driven Development}}
    {
        Desenvolvimento orientado a comportamento.
        
        \TODO{Melhor descrição para BDD.}
    }
    
\novoacronimo
    {IA} %label
    {IA} %symbol
    {{Interpretação Abstrata}} %name
    {
        A Interpretação Abstrata, geralmente abreviada como IA (ou AI na 
        literatura internacional), é uma teoria que formaliza aproximações 
        plausíveis da semântica de programas e sistemas computacionais. Mais 
        especificamente, trata do problema de obter informações sobre a 
        semântica de programas através de execuções parciais, não 
        necessariamente em tempo de execução.
    }

\novoacronimo
    {GPGPU}
    {GPGPU}
    {{General Purpose Graphics Processing Unit}}
    {
        Unidade de processamento gráfica de uso geral.
                
        \TODO{Melhor descrição para GPGPU.}
    }


\novoacronimo
    {OpenCL}
    {OpenCL}
    {{Open Computing Language}}
    {
        Open Computing Language.
                
        \TODO{Melhor descrição para OpenCL.}
    }
    
\novoacronimo
    {AE} %label
    {AE} %symbol
    {{Análise Estática}} %name
    {
        Análise estática.
        
        \TODO{Melhor descrição do termo AE.}
    }

\novoacronimo
    {TI}
    {TI}
    {{Tecnologia da Informação}}
    {
        Em uma forma mais ampla, pode-se definir a Tecnologia da 
        Informação como o conjunto de técnicas, atividades e soluções voltadas 
        à gerência, geração, uso, acesso, segurança e transmissão de dados e 
        informação por meios computacionais.
    }

\novoacronimo
    {STR}
    {STR}
    {{Sistema de Tempo Real}}
    {
        Sistemas nos quais há exigências temporais acerca da realização de suas 
        tarefas. Não se trata de executar as tarefas no menor tempo possível, 
        mas sim de executá-las \emph{dentro do tempo disponível}.
    }

% ---
% D E F I N I Ç Õ E S
% ---

\novadefinicaopl
    {sistema_critico}
    {sistema crítico}
    {sistemas críticos}
    {
        Vide seção \emph{\ref{introducao_sobre_criticidade}} (página 
        \pageref{introducao_sobre_criticidade}).
    }

\novadefinicaopl
    {dominioAbstrato}
    {domínio abstrato}
    {domínios abstratos}
    {
        Domínio abstrato.
        
        \TODO{Melhor descrição para domínios abstratos.}
    }

\novadefinicao
    {dependabilidade}
    {dependabilidade}
    {
        Termo que representa uma abordagem mais abrangente, em relação à 
        caracterização da criticidade de sistemas. Explicando de forma sucinta, 
        entende-se que a \emph{dependabilidade} de um sistema não 
        seja definida apenas pelos seus aspectos técnicos, assim como 
        ocorre com termos como \emph{confiabilidade}, muito usados 
        hoje. \citeauthor{rushby_critical_1994} cita o trabalho de 
        \citeonline{laprie_dependable_1985} para definir o conceito 
        de \emph{dependabilidade}:
        \blockquote{\textit{
            O termo \emph{"dependabilidade"} é usado para escapar de 
            significados técnicos especializados que hoje estão 
            associados à termos como \emph{"confiabilidade"}. Também almeja-se 
            ter um termo para uma abordagem que, no geral, possa adotar, 
            incluir e unificar muitos problemas e técnicas que têm geralmente 
            sido consideradas isoladamente -- como por exemplo, tolerância a 
            falhas, confiabilidade, corretude, proteção, capacidade de 
            sobrevivência e segurança.} \cite{rushby_critical_1994}
        }
    }

\novadefinicao
    {protecao}
    {proteção contra falhas}
    {
        O termo original em inglês é \emph{safety}, que significa 
        \emph{segurança}, se traduzido literalmente para o português. No 
        entanto, há uma conotação forte do termo \emph{segurança} para questões 
        que envolvem a \emph{segurança da informação}. Na literatura estudada, 
        \emph{safety} possui um sentido diferente, mais voltado à 
        \emph{segurança contra falhas}; sobre garantias de integridade do 
        sistema, de pessoas, do meio ambiente, de propriedades e de ativos após 
        a ocorrência de uma falha.
        
        Infelizmente, não foi encontrado nenhum termo na literatura brasileira 
        com este significado específico. O uso desta expressão fica, portanto, 
        exclusiva à esta monografia, apenas para explicitar a diferença em 
        relação às terminologias que envolvem segurança da informação.
    }

\novadefinicaopl
    {conjunto}
    {conjunto}
    {conjuntos}
    {
        Um conjunto é, em sua definição mais simples, uma coleção bem definida 
        de elementos.
    }

\novadefinicaopl
    {conjuntoOrdenado}
    {conjunto ordenado}
    {conjuntos ordenados}
    {
        Um \gls{conjunto}, definido por $(E, \rel)$, onde $E$ são os elementos 
        e $\rel$ é um operador de relação entre os elementos, é dito 
        \emph{ordenado} quando, para três elementos $a, b, c \in E$ ele atende 
        a três propriedades: \textbf{reflexividade} ($a \rel a$), 
        \textbf{antissimetria} (se $a \rel b$ e $b \rel a$, então $a = b$) e 
        \textbf{transitividade} (se $a \rel b$ e $b \rel c$, então $a \rel c$).
        
        A questão da ordenação parcial ou total é em relação à aplicabilidade
        do operador de relação a todos ou apenas alguns dos elementos do 
        conjunto. Quando um conjunto é parcialmente ordenado, chamamos-o de 
        \gls{poset}.
    }

\novadefinicaopl
    {poset}
    {poset}
    {posets}
    {
        Do inglês \emph{partially-ordered sets}, ou conjuntos parcialmente 
        ordenados. Veja \gls{conjuntoOrdenado}.
    }

\novadefinicaopl
    {reticulado}
    {reticulado}
    {reticulados}
    {
        Um \gls{poset} $(L, \rel)$ é chamado de reticulado se todo par
        de elementos ${a, b}$ de $L$ possui tanto uma menor cota superior de 
        elementos, (LUB, \emph{lowest upper bound}, supremo) como uma maior 
        cota inferior (GLB, \emph{greatest lower bound}, ínfimo).
    }