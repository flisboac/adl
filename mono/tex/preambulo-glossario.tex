
\newglossaryentry{pi}
{
  name={\ensuremath{\pi}},
  description={ratio of circumference of circle to its diameter},
  sort=pi,
  type=\symboltype
}

\newglossaryentry{pi2}
{
  name={\ensuremath{\pi}2},
  description={ratio of circumference of circle to its diameter},
  sort=pi2,
  type=\symboltype
}

\newacronym[
    see={[Vide:]{gls-IA}}
    ]{IA} % label
    {IA} % name
    {Interpretação Abstrata} % long name
\newglossaryentry{gls-IA}{
    type=\glossarytype,
    name={Interpretação Abstrata},
    description={
        A Interpretação Abstrata, geralmente abreviada como IA (ou AI na 
        literatura internacional), é uma teoria que formaliza aproximações 
        plausíveis da semântica de programas e sistemas computacionais. Mais 
        especificamente, trata do problema de obter informações sobre a 
        semântica de programas através de execuções parciais.
    }
}
\newglossaryentry{acr-IA}{
    type=\glossarytype,
    name={IA},
    see={gls-IA},
    description={}
    %description={\glssee[Veja:]{acr-IA}{gls-IA}}
    %description={Vide \glsentrytext{gls-IA}}
}

