
\chapter[Introdução]{Introdução}

A sociedade de hoje é cada vez mais dependente da tecnologia. É possível ver 
exemplos do uso da computação e da \gls{TI} para solucionar problemas diversos, 
que vão desde simples aplicativos pessoais até complexos sistemas 
informatizados à nível internacional.

A tecnologia está disponível para facilitar nossas vidas, mas dado o seu papel 
cada vez mais crucial na vida das pessoas, torna-se necessário garantir a 
segurança, confiabilidade e corretura dos sistemas e softwares dos quais 
dependemos: quanto mais crítico, mais deseja-se garantir uma margem de 
confiança em relação à ausência ou resiliência a erros. Não 
somente, pode se tornar necessário também, na ocorrência de um evento adverso, 
implementar meios de recuperação ou contingência.

\section{Sobre sistemas de software}

    Diversos autores versam sobre teorias e práticas envolvendo o planejamento, 
    execução e operação segura de sistemas em vários âmbitos, e se utilizando 
    de diversas visões sobre o tema -- que muitas das vezes, se complementam. É 
    comum 

    \subsection{Definindo criticidade e falhas} 
    \label{introducao_sobre_criticidade}

    \citeonline{sommerville_software_2011} define um \gls{sistema_critico} como 
    \emph{"um sistema de computador no qual falhas podem resultar em perdas 
    significativas, seja no âmbito humano, econômico ou ambiental."} Na mesma 
    obra, são caracterizados, ao longo dos capítulos, três tipos de sistemas 
    (uma lista mais sucinta pode ser encontrada em 
    \cite{sommerville_critical_2008}):
    
    \begin{description}
        \item[Sistemas de segurança crítica:] \emph{"Um sistema no qual falhas 
        possam resultar em ferimentos físicos, perda de vidas ou sérios danos 
        ambientais." Exemplo: um sistema de monitoramento e controle de uma 
        usina nuclear.}
        
        \item[Sistemas de missão crítica:] \emph{"Um sistema no qual falhas 
        possam causar o insucesso de alguma atividade crucial baseada em 
        objetivos." Exemplo: um sistema de navegação e controle aéreo.}
        
        \item[Sistemas de negócio crítico:] \emph{"Um sistema no qual falhas 
        possam resultar em altíssimos custos a um negócio que utiliza tal 
        sistema." Exemplo: um sistema de troca de ações na bolsa de valores. }
    \end{description}
    
    Esta taxonomia é simples, direta e amplamente aplicável, mas não é 
    universal. Também não é definido com clareza o conceito de falha. Ademais, 
    necessidades específicas exigem modelos distintos e uma visão de 
    criticidade que se adeque melhor ao contexto no qual o sistema está 
    inserido. Talvez por isto que a literatura existente acerca de sistemas 
    críticos seja tão rica e abrangente em terminologia e conteúdo. 
    \citeonline{rushby_critical_1994} explorou alguns modelos de análise e 
    categorização de outros autores, no tema de \glspl{sistema_critico}, de uma 
    forma bem objetiva e clara. Quatro destes modelos são descritos, em termos 
    da confiabilidade atribuída à um sistema:
    
    \begin{description}
        \item[Pela \Gls{dependabilidade}:] Um sistema confiável 
        (segundo o presente conceito) é \emph{"aquele no qual a dependência 
        pode ser justificadamente colocada em determinados aspectos da 
        qualidade do serviço que ele entrega. A qualidade de um serviço inclui 
        tanto a sua corretura (e.g. conformidade com os requisitos, 
        especificações e expectativas) e a continuidade de sua entrega."}. Ou 
        seja, o sistema é confiável se ele implementa perfeitamente a sua 
        especificação, atende à todos os seus requisitos e entrega seus 
        serviços com garantida qualidade, independente das consequências de uma 
        eventual falha.
    
        \item[Pela \emph{Engenharia de \Gls{protecao}}:] \emph{"A 
        confiabilidade de um sistema diz respeito à incidência de falhas; já a 
        \gls{protecao} se importa com a ocorrência de acidentes ou 
        acontecimentos adversos -- eventos não planejados que resultam em 
        mortes, injúrias, doenças, danos à (ou perda de) propriedade ou danos 
        ao ambiente."} As falhas de sistema são definidas em termos dos 
        serviços e da descrição deste sistema, enquanto que a \gls{protecao} é 
        definida em termos das consequências externas ao sistema e à forma como 
        ele reage à elas. Pode-se ter um sistema perfeitamente confiável em 
        termos da adequação aos seus requisitos, mas igualmente inseguro em 
        relação ao tratamento e consequências de seus erros.
        
        \item[Pensando em \emph{Sistemas Seguros}:] Esta abordagem tem
        uma visão voltada à \emph{segurança da informação}, e a confiabilidade 
        é descrita em termos de propriedades que envolvem garantias de 
        privacidade e confidencialidade.
        
        \item[Pensando em \gls{STR}:] \emph{"Um \gls{STR} é aquele cuja 
        corretura depende não somente dos valores das suas saídas, mas também 
        do tempo utilizado para a produzi-las. (...) Há dois problemas 
        importantes sobre o desenvolvimento de um \gls{STR}: a derivação de 
        restrições temporais, e a construção de uma estrutura de sistema (mais 
        particularmente um regime de agendamento) que garanta a satisfação 
        destas restrições."} A literatura descreve formas diversas de 
        qualificação e tipificação de um \gls{STR}, mas em um âmbito geral, as 
        consequências do não cumprimento das restrições temporais de um 
        \gls{STR} podem ser desastrosas, pois as aplicações baseadas em um 
        \gls{STR} comumente atuam em ambientes críticos e que exigem respostas 
        imediatas e tempos de execução minimamente delimitados (e.g. sistemas 
        de controle de vôo, sistemas de automação, sistemas de monitoramento de 
        saúde).
    \end{description}
    
    De acordo com os modelos de \citeauthor{rushby_critical_1994}, o conceito 
    de \emph{falha de sistema} pode variar de acordo com a metodologia de
    qualificação um sistema, mas de uma forma geral, podemos concluir que uma 
    falha é \emph{qualquer evento adverso segundo as descrições do sistema}, 
    sejam elas previstas ou simplesmente desvios à especificação. Falhas podem 
    ser tratáveis (e.g. falha ao iniciar uma conexão a um banco de dados) ou 
    não (e.g. queda de energia); já a \emph{gravidade} de uma falha está 
    relacionada com as suas \emph{consequências} (e.g. uma queda de energia em 
    um aparelho médico poderia custar uma vida, mas a reconexão ao banco de 
    dados poderia ser retentada em um momento posterior).
    
    Para efeitos de discussão do que vem a seguir, utilizaremos as definições 
    de sistemas críticos assim como descritos em 
    \citeonline{sommerville_software_2011} e as conclusões acerca de falhas de 
    software extraídas com base na leitura de \citeonline{rushby_critical_1994}.
    
    
    \subsection{Consequências das falhas de software}
    
    A não adoção de conceitos e práticas que garantam estas características de 
    segurança minimamente necessárias à softwares críticos ou de 
    grande volume tem consequências muitas vezes desastrosas. 
    Em \citeonline{wong_recent_2010} e \citeonline{wong_role_2009}, são 
    listados alguns dos acidentes mais trágicos ou danosos envolvendo sistemas 
    críticos, onde o software teve participação ou culpa sobre o ocorrido.
    
    \TODO{Tradução e referências extras de todos os itens.}
    \begin{description}
        \item[O Caso Patriot.] Arábia Saudita, 25 de Fevereiro de 1991. 
        
        
        \item[Panes no Aeroporto de Los Angeles.] Los Angeles, Setembro de 
        2004.
        
        \item[Dosagens erradas de radiação.] Instituto Nacional de 
        Oncologia do Panamá, Março de 2011.
        
        \item[Falta de energia nos EUA e Canadá.] Nordeste dos EUA e Sudeste do 
        Canadá, 14 de Agosto de 2003.
    \end{description} 
    
    Casos como os citados anteriormente nos alertam para a importância de se 
    adotar uma prática ou metodologia de prevenção e mitigação de falhas, 
    tenham elas origem em software ou não. \cite{sommerville_critical_2008} 
    cita uma série de métodos e dá exemplos excelentes sobre a aplicação deles.


    \subsection{A importância dos métodos formais}
     
    \TODO{Ver capítulo e página onde eu li isso: 
    \citeonline{sommerville_critical_2008} estima que 50\% do custo para o 
    desenvolvimento de software vem dos esforços para garantir a qualidade e 
    adequação}
    
    \TODO{Buscar referências: Métodos formais podem reduzir custos e dar 
    garantias que não podem ser totalmente provadas por métodos de teste 
    tradicionais.}
    
    \TODO{Importância do domínio dos octógonos: Mostrar o uso do domínio dos 
    octógonos no mercado, e como ele é melhor em relação à outros domínios para 
    determinados tipos de análise estática/interpretação abstrata.}
    
    
    \section{A relevância da GPGPU}
    
    \TODO{Presença das GPUs no mercado: smartphones, computadores, etc. 
    Apresentar gráficos.}
    
    \TODO{Importância do GPGPU: como empresas estão usando plataformas de GPGPU 
    (CUDA/OpenCL/DirectCompute/RenderScript Compute) para implementar 
    inteligência artificial.}
    
    \TODO{GPGPU na ciência: Como o OpenCL/CUDA/GPGPU tem ganhado espaço na 
    ciência.}
    
    \TODO{Como o GPGPU é subutilizado para implementação de ferramentas de 
    Interpretação Abstrata.}
    
    
    \section{Objetivo do trabalho}
    
    O objetivo deste trabalho é implementar os operadores e algoritmos do 
    domínio dos octógonos, primariamente em \gls{GPU} utilizando o 
    \gls{OpenCL}, mas também em \gls{CPU}. Almeja-se fazer uma comparação de 
    tempo de execução e uso de recursos entre implementações dos algoritmos em 
    \gls{CPU} e em \gls{GPU} com diversos tamanhos de problema (e.g. número de 
    variáveis, número de restrições), e avaliar a possibilidade e casos em que 
    o uso da \gls{GPU} se justifique.
    
    O produto final do projeto será uma biblioteca, que funcionará tanto em 
    \gls{CPU} quanto em \gls{GPU}, e que terá interfaces em C++ para que seja 
    possível a sua utilização de forma embarcada em outros softwares (e.g. 
    processos, aplicações, serviços, outras bibliotecas, outras linguagens de 
    programação). O projeto fará uso de técnicas de desenvolvimento e testes 
    como o \gls{TDD} e o \gls{BDD}, para garantir a corretura das 
    implementações dos algoritmos.
    
    Sabe-se que a publicação seminal para o domínio dos octógonos é a da tese 
    de \citeonline{mine_weakly_2004}, que foi a obra que definiu pela primeira 
    vez este domínio abstrato. No entanto, para este trabalho, o embasamento 
    teórico virá primariamente de \citeonline{chawdhary_simple_2014}, pois são 
    descritas otimizações que serão importantes para o bom funcionamento dos 
    algoritmos em \gls{GPU} em alguns casos.
    
    
        \subsection{Justificativa}
    
        Este projeto foi iniciado na Universidade de Kent, Inglaterra, como um 
        projeto de pesquisa a ser realizado no período das férias de verão de 
        2014. No entanto, mesmo com o bom andamento, o projeto não foi 
        finalizado até o fim do período. O objetivo era utilizá-lo para criar 
        ferramentas de análise estática, e assim verificar falhas de segurança 
        em arquivos executáveis em formatos de máquina (binários); faria parte 
        da pesquisa do pós-doutorando Aziem Chawdhary \footnote{Website: 
        \url{http://www.chawdhary.co.uk/about.html}}, e baseia-se em um artigo 
        escrito por ele e seu orientador.
    
        De qualquer forma, há boa oportunidade para o uso do produto final (a 
        biblioteca) do projeto em outras aplicações. Há implementações dos 
        domínios dos octógonos em \gls{CPU}, mas não se tem nenhum relato da de 
        pesquisas ou implementações dos algoritmos em \gls{CPU}. Dada a 
        relevância e presença cada vez maiores de plataformas como o 
        \gls{GPGPU} -- não só no mercado mas também no meio científico -- e a 
        adequação dos algoritmos para o paralelismo, pesquisas na área podem 
        trazer melhorias. Em especial, a verificação das condições de grandes 
        sistemas em tempo de execução pode se utilizar de técnicas de \gls{IA} 
        e paralelismo -- o que pode, inclusive, ser um tema a se explorar no 
        futuro.
    
    
    \section{Organização do trabalho}
    
    Esta introdução tem como intuito demonstrar a importância da verificação, 
    validação e garantias da segurança de sistemas críticos. Também foram 
    demonstradas algumas das ferramentas e metodologias utilizadas para se 
    atingir este objetivo, com uma atenção especial dada à \gls{IA} -- mais 
    especificamente, ao domínio dos octógonos -- e às formas nas quais ela é 
    utilizada em aplicações reais.
    
    No capítulo \ref{fundamentacao_ia}, o campo de \gls{IA} e a literatura 
    acerca do domínio dos octógonos será exposta em maiores detalhes. Já no 
    capítulo \ref{fundamentacao_cl}, serão introduzidos os conceitos básicos de 
    paralelismo computacional, bem como a descrição, arquitetura e exemplos de 
    uso do \gls{OpenCL}. Estes dois capítulos servem como a fundamentação 
    teórica e prática para os que seguem.
    
    No capítulo \ref{metodologia}, será exposta a metodologia de pesquisa, 
    organização do projeto e outras informações importantes que pautarão a 
    discussão do desenvolvimento do projeto como um todo.
    
    No capítulo \ref{resultados} serão apresentados os resultados obtidos com 
    os testes, que serão utilizados no capítulo \ref{conclusao} para extrair as 
    conclusões finais.
    
    Este documento fará uso de terminologia altamente técnica ao longo do seu 
    desenvolvimento. Havendo qualquer dúvida acerca de um termo, sigla ou 
    símbolo, basta referir-se à:
    
    \begin{itemize}
        \item Lista de Símbolos, na página \pageref{pre:simbolos};
        \item Lista de Abreviaturas, na página \pageref{pre:abreviaturas}; ou
        \item Glossário, na página \pageref{pos:glossario}.
    \end{itemize} 