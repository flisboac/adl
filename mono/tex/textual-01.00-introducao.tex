
\chapter[Introdução]{Introdução}

A sociedade de hoje é cada vez mais dependente da tecnologia. É possível ver 
exemplos do uso da computação e da \gls{TI} para solucionar problemas diversos, 
que vão desde simples aplicativos pessoais até complexos sistemas 
informatizados à nível internacional.

A tecnologia está disponível para facilitar nossas vidas, mas dado o seu papel 
cada vez mais crucial na vida das pessoas, torna-se necessário garantir a 
segurança, confiabilidade e corretura dos sistemas e softwares dos quais 
dependemos: quanto mais crítico, mais deseja-se garantir uma margem de 
confiança em relação à ausência ou resiliência a erros. Não 
somente, pode se tornar necessário também, na ocorrência de um evento adverso, 
implementar meios de recuperação ou contingência. Diversos autores versam sobre 
teorias e práticas envolvendo o planejamento, execução e operação segura de 
sistemas em vários âmbitos, e se utilizando de diversas visões sobre o tema -- 
que muitas das vezes, se complementam.

\section{Sobre sistemas de software}

    \subsection{Definindo criticidade e falhas} 
    \label{introducao_sobre_criticidade}

    \citeonline{sommerville_software_2011} define um \gls{sistema_critico} como 
    \emph{"um sistema de computador no qual falhas podem resultar em perdas 
    significativas, seja no âmbito humano, econômico ou ambiental."} Na mesma 
    obra, são caracterizados, ao longo dos capítulos, três tipos de sistemas 
    (uma lista mais sucinta pode ser encontrada em 
    \cite{sommerville_critical_2008}):
    
    \begin{description}
        \item[Sistemas de segurança crítica]
        \item[Sistemas de missão crítica]
        \item[Sistemas de negócio crítico]
    \end{description}
    
    No entanto, esta taxonomia não é universal. Necessidades específicas exigem 
    modelos distintos; talvez por isto que a literatura existente acerca de 
    sistemas críticos seja tão abrangente. \citeonline{rushby_critical_1994} 
    explorou estes modelos de análise e categorização de 
    \glspl{sistema_critico} de uma forma bem objetiva e clara, e elencou quatro 
    deles:
    
    \begin{description}
        \item[Pela \Gls{dependabilidade}] Um sistema confiável 
        (segundo o presente conceito) é \emph{"aquele no qual a dependência 
        pode ser justificadamente colocada em determinados aspectos da 
        qualidade do serviço que ele entrega. A qualidade de um serviço inclui 
        tanto a sua corretude (e.g. conformidade com os requisitos, 
        especificações e expectativas) e a continuidade de sua entrega."}. Ou 
        seja, um sistema que implementa perfeitamente a sua especificação, 
        atende à todos os seus requisitos (funcionais e não-funcionais), e 
        entrega seus serviços com garantida qualidade, então ele é um sistema 
        confiável.
    
        \item[Pela \emph{Engenharia de \Gls{protecao}}] \emph{"A 
        confiabilidade de um sistema diz respeito à incidência de falhas; já a 
        \gls{protecao} se importa com a ocorrência de acidentes ou 
        acontecimentos adversos -- eventos não planejados que resultam em 
        mortes, injúrias, doenças, danos à (ou perda de) propriedade ou danos 
        ao ambiente. Enquanto que falhas de sistema são definidas em termos dos 
        serviços deste sistema, a \gls{protecao} é definida em termos das 
        consequências externas ao sistema. Se os serviços de sistema são 
        especificados incorretamente, então o sistema pode ser inseguro, apesar 
        de perfeitamente confiável."}.
        
        \item[Pensando em \emph{Sistemas Seguros}] Esta abordagem tem
        uma visão voltada à \emph{segurança da informação}, e a confiabilidade 
        é descrita em termos de propriedades envolvendo garantias de 
        privacidade e confidencialidade.
        
        \item[Pensando em \gls{STR}] \emph{"Um \gls{STR} é aquele cuja 
        corretura depende não somente dos valores das suas saídas, mas também 
        do tempo utilizado para a produzi-las. (...) Há dois problemas 
        importantes sobre o desenvolvimento de um \gls{STR}: a derivação de 
        restrições temporais, e a construção de uma estrutura de sistema (mais 
        particularmente um regime de agendamento) que garanta a satisfação 
        destas restrições."}
    \end{description}
    
    De acordo com os modelos de \citeauthor{rushby_critical_1994}, o conceito 
    de \emph{falha de sistema} pode variar de acordo com a forma como se 
    qualifica um sistema, mas de uma forma geral, uma falha é \emph{qualquer 
    evento adverso, e leva em consideração tão somente as descrições do 
    sistema}. Falhas podem ser tratáveis (e.g. falha ao iniciar uma conexão a 
    um banco de dados) ou não (e.g. queda de energia); já a \emph{gravidade} de 
    uma falha está relacionada com as suas \emph{consequências} (e.g. uma queda 
    de energia em um aparelho médico poderia custar uma vida, mas a reconexão 
    ao banco de dados poderia ser retentada em um momento posterior). É por 
    estes motivos, e para evitar prejuízos, que se torna prudente adotar uma 
    prática ou metodologia de prevenção e mitigação de erros.    
    \cite{sommerville_critical_2008} cita uma série de métodos e dá exemplos 
    excelentes sobre a aplicação deles

    
    \subsection{Consequências das falhas de software}
    
    A não adoção de conceitos e práticas que garantam estas características de 
    segurança minimamente necessárias à softwares de missão crítica ou de 
    grande volume tem consequências muitas vezes desastrosas.
    
    \subsection{Dificuldades em garantir a ausência de erros}
    
    (Ver onde eu li isso: Sommerville estima que 50\% do custo para o 
    desenvolvimento de software vem dos esforços para garantir a qualidade e 
    adequação)

    \subsection{A importância dos métodos formais}
        
    \section{Objetivo}
    \section{Justificativa}
    \section{Organização do trabalho}
    Citação normal: \cite{chawdhary_simple_2014}. Citação online: \citeonline{chawdhary_simple_2014}.