
% ---
% RESUMOS
% ---

% resumo em português
\setlength{\absparsep}{18pt} % ajusta o espaçamento dos parágrafos do resumo
\begin{resumo}

O campo de Interpretação Abstrata (IA) avançou muito desde a sua concepção nos 
anos 70. A IA é usada para aproximar o comportamento e modelar sistemas formais 
e computacionais. Ferramentas baseadas em conceitos de IA são usadas 
constantemente em \emph{frameworks} de análise estática de programas, para 
realizar tarefas diversas como prever defeitos, provar a corretura ou ausência 
de bugs, encontrar falhas de segurança e inferir tipos de dados em ambientes de 
linguagens de programação. Um conceito crucial na área de IA é o do domínio 
abstrato. Muitos domínios numéricos foram criados e usados com sucesso para 
resolver problemas reais, como os domínios dos intervalos, das congruências e 
dos polígonos.

Um deles, no entanto, tem recebido destaque recentemente: o domínio dos 
octógonos. Criado em 2006 por Antoine Miné, é um domínio abstrato fracamente 
relacional, que merece destaque pela sua baixa complexidade computacional e 
pela razoável adequação do algoritmo ao paralelismo. Analisadores estáticos já 
foram criados com base neste domínio abstrato, e usados com sucesso, mas poucos 
deles até hoje exploraram a possibilidade de implementar os operadores 
octogonais de forma paralela em GPU. Considerando o custo cada vez mais baixo 
de placas de vídeo e de plataformas de GPGPU (General Processing GPU), torna-se 
relevante a exploração de possíveis implementações e adaptações dos algoritmos 
dos domínios dos octógonos para esta nova realidade.

O objetivo deste trabalho é implementar os algoritmos e operadores do domínio 
dos octógonos em GPU, utilizando o OpenCL. Comparações serão realizadas com 
outras duas implementações baseadas em CPU e SSE, a fim de concluir a adequação 
dos algoritmos e decidir em quais tipos de ocasião o uso de computação paralela 
em GPU se torna mais apropriado.

\vspace{\onelineskip}

% TODO: Criar um comando \keywords para as palavras-chave.
\textbf{Palavras-chave}: Interpretação Abstrata. Domínio dos octógonos. GPGPU. Computação Paralela.
\end{resumo}

% resumo em inglês
\begin{resumo}[Abstract]
 \begin{otherlanguage*}{english}
   This is the english abstract.

   \vspace{\onelineskip}
 
   \noindent 
   \textbf{Keywords}: latex. abntex. text editoration.
 \end{otherlanguage*}
\end{resumo}

